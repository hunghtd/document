\documentclass[10pt]{memoir}

% based on kieran healy's memoir modifications
\usepackage{mako-mem}
\chapterstyle{article-3}
\pagestyle{memo}

\usepackage{ucs}
\usepackage[utf8x]{inputenc}

\usepackage[T1]{fontenc}
\usepackage{textcomp}
\usepackage[garamond]{mathdesign}

\usepackage[letterpaper,left=1.2in,right=1.2in,top=1.2in,bottom=1.2in]{geometry}

% packages i use in essentially every document
\usepackage{graphicx}
\usepackage{wrapfig}
\usepackage{enumerate}

% packages i use in many documents but leave off by default
% \usepackage{amsmath, amsthm, amssymb}
% \usepackage{dcolumn}
% \usepackage{endfloat}

% import and customize urls
\usepackage[usenames,dvipsnames]{color}
\usepackage[breaklinks]{hyperref}

\hypersetup{colorlinks=true, linkcolor=Black, citecolor=Black, filecolor=Blue,
    urlcolor=Blue, unicode=true}

% add bibliographic stuff 
% \usepackage[round]{natbib}
% \def\citepos#1{\citeauthor{#1}'s (\citeyear{#1})}
% \def\citespos#1{\citeauthor{#1}' (\citeyear{#1})}

% import vc stuff after running `make vc`: \input{vc} \pagestyle{kjhgit}

\begin{document}

\setlength{\parskip}{4.5pt}

\baselineskip 14.5pt

\title{Research Statement}
\author{Benjamin Mako Hill}

\maketitle

My research is focused on collective action in online communities and
seeks to understand why some attempts at collaborative production --
like Wikipedia and Linux -- build large volunteer communities while
the vast majority never attract even a second contributor. I am
particularly interested in how the design of communication and
information technologies shape social outcomes like the decision to
join a community or contribute to a public good. My research is deeply
interdisciplinary and lies at the intersection of sociology,
communication, and human-computer interaction. I analyze data from
online communities that make failures of collective action newly
visible with ``big data'' research methods from software engineering
to answer fundamental social scientific questions.

Seeking to understand the determinants of collective action, my
research has been driven by three overlapping themes: (1)
population-level observational studies comparing failures to build
communities to the rare successful attempts; (2) attention to the role
of reputation and status in the mobilization of volunteers; and (3)
analyses of design changes as ``natural experiments'' to build a
deeper, and often causal, understanding of social processes from
observational data. Nearly all of my work incorporates at least two of
these themes.

\section{Population-Level Observational Studies}

Although there have been many thousands of studies of online
collective action and peer production, the vast majority of these
studies have only considered successful projects like Wikipedia and
GNU/Linux.  The majority of research on collective action -- online
and off -- has only considered projects that have successfully
mobilized. In this sense, most previous analyses of
collection action have systematically selected on their dependent
variable. Most of my research treats projects as the unit of analysis
and mobilization as the dependent variable to compare successful
examples of collective actions to failures.

% \begin{wrapfigure}{r}{0.4\textwidth}
%  \begin{centering}
%  \includegraphics[width=2.4in]{figures/wp_citations_by_year.png}
%  \caption{Number of published academic articles with ``wikipedia''
%  in title by year.}
%  \label{fig:wppapers}
%  \end{centering}
%\end{wrapfigure}

\begin{wrapfigure}{r}{2.6in}
 \begin{centering}
 \includegraphics[width=2.6in]{figures/scratch_screenshot_default.png}
 \caption{A screenshot of the Scratch programming environment
   where users create animations and interactive games.}
 \label{fig:scratchapp}
 \end{centering}
 \vspace{-2em}
\end{wrapfigure}


For example, in a working paper that is part of my dissertation, I
compare Wikipedia to seven attempts to create online collaborative
encyclopedia projects that were launched before Wikipedia
\cite{hill_almost_2012}. Using an inductive, grounded-theory based
analysis of founder interviews and archival data, I propose four
hypothesis to explain why Wikipedia attracted many more contributors
than similar projects. Although the paper's methods diverge from the
quantitative, ``big data'' approach typical of most of my work, the
research question and strategy is representative.

I have also followed this strategy in a series of quantitative
studies of the Scratch online community: a public website with a large
community of users who create, share, and remix interactive media. The
community is built around the Scratch programming environment: a
freely downloadable desktop application that allows amateur creators
to combine media with programming code (see Figure
\ref{fig:scratchapp}). Despite the fact that Scratch is a community
designed to promote collaboration through content remixing, only about
ten percent of Scratch projects attract a second
contributor.

\begin{wrapfigure}{l}{2.6in}
 \begin{centering}
 \includegraphics[width=2.6in]{figures/frontpage_modified-topremix.png}
  \caption{The front page of the Scratch online community where users
    can share and collaborate on projects.}
 \label{fig:scratchfrontpage}
 \end{centering}
 \vspace{-2em}
\end{wrapfigure}

In one study, forthcoming in American Behavioral Scientist, I test
several of the most commonly cited theories associated with
``generativity'' (i.e., qualities of technology or content that make
some works more fertile ground for collaboration). I find some support
for previous theories but also find that, across the board, factors
associated with increased collaboration are also associated with less
original and transformative modes of joint-work
\cite{hill_remixing_2012}. In another study of Scratch, I show that
more superficial collaboration leads to negative reactions and
community displeasure \cite{hill_responses_2010}.

I am conducting a similar population-level analysis in a new dataset I
have created for my dissertation that includes 80,000 public attempts
at wikis (i.e., public, editable, websites similar to Wikipedia). In
my first working paper using this dataset, I consider
inter-organizational effects of competition for volunteer labor and
find little support for a widely cited ecological model of collective
action from sociology that treats volunteer labor as fixed and finite
resource. Instead, I show that contributions to different wikis on the
same topic or theme are driven primarily by environment-level changes
in interest and that projects can even benefit from complimentarities
and synergies \cite{hill_is_2012}.

\section{Reputation and Status}

Although empirical research comparing successful and unsuccessful peer
projects has been rare, theories have been widespread. No theory has
been more influential than the suggestion that, in the absence of
pecuniary rewards, contributions to online public good are driven by
the possibility of increased reputation and status conferred upon
contributors.

\begin{wrapfigure}{r}{0.3\textwidth}
 \vspace{-1em}
 \begin{centering}
 \includegraphics[width=1.9in]{figures/barnstar_alone.png}
 \caption{Image of a ``barnstar'' social award given by Wikipedia
   contributors to each other to recognize positive contributions .}
 \label{fig:barnstar}
 \end{centering}
 \vspace{-1em}
\end{wrapfigure}

In a study of status-based awards in Wikipedia called ``barnstars''
(see Figure \ref{fig:barnstar}) that I will be submitting to a major
sociology journal by the end of this year, I provide an empirical test
of an influential status-based theory of collective action from
sociology. Although the study finds support for the widely
hypothesized ``virtuous cycle'' of status rewards both causing and
being caused by contributions, it also finds that this effect is
limited to a sub-population of contributors to Wikipedia -- i.e.,
those who show off their awards \cite{hill_status_2012}. This result
has broad implications for both status-based theories of collective
action as well the design of reputation-based rewards.

In a mixed methods study of Scratch, nominated for best paper at the
CHI 2011 conference \cite{monroy-hernandez_computers_2011}, I
presented both a quantitative analysis of a design change and in-depth
interviews of users to demonstrate how credit-giving is ineffective
when it stems from an automated system because systems fail to
reinforce status-ordering with credible human expressions of social
deference and gratitude.

\section{Design-Driven Natural Experiments}

\begin{wrapfigure}{r}{0.25\textwidth}
 \begin{centering}
 \includegraphics[width=1.5in]{figures/lilypad.png}
 \caption{A image of the LilyPad Arduino microcontroller.}
 \label{fig:lilypad}
 \end{centering}
\end{wrapfigure}

Although nearly all of my work has important implications for the
design of socio-technical systems, I have structured much of my work
around the evaluation of technological design changes. In several papers, I
treat design changes as ``natural experiments'' that
exogenously change the ways that social structure is enacted in order
to both build causal understanding of social phenomena from field data and to tighten the
distance between theory and and design.

For example, to evaluate the impact of status-based incentives and
collaboration in Scratch, I use a regression discontinuity framework
to measure the causal effect of increased status for collaboration
\cite{hill_causal_2012}. In that study, which I am preparing for
submission to a communication journal this fall, I show that
highlighting collaborative projects on the Scratch web page (see the
bottom of Figure \ref{fig:scratchfrontpage}) resulted in more
collaboration but also caused a decrease in the amount of total effort
exerted by contributors. Speaking to fundamental sociological work in
the literature on collective action, I present evidence that this
decrease is driven by both an the influx of new contributors and a
decrease in the effort and contributions of established participants.

In other papers, I have helped analyze sales records of hobbyist
microcontrollers to suggest that relatively simple design changes in
the \emph{LilyPad Arduino} -- a electronics toolkit minimally
re-designed for women and girls (see Figure \ref{fig:lilypad}) -- can
lead to large increases in the proportion of women contributors and
drastic shifts in the type of projects created
\cite{buechley_lilypad_2010}. In other work, I have explored how
technical errors may be able to provide similar opportunities for
analysis \cite{hill_revealing_2010}.

% or changes in socio-technical systems describing responsibility for a piece of software can lead to an important impact in the type and structure of contributions in peer production \cite{michlmayr_quality_2003}

\section{Research Agenda}

My research agenda involves further exploration of the determinants of
collection action online -- especially using a series of large new
datasets I have assembled for my dissertation. I plan to both continue
on this research trajectory and to create new social and technical
infrastructure that will allow others researchers to join me in ``big
data'' observational research in active communities. This section
outlines some future directions I plan to explore.

\emph{Toolkits for Experimental Social Design} -- My research has been
possible through personal relationships I have with a series of
organizations with large active online communities (e.g., the MIT
Media Lab and the Wikimedia Foundation). These organizations, like
many others, make design changes to the software that supports their
communities to encourage contributions and improve aspects of their
users' experiences. Most of the time, these organizations have very
little idea if these changes are effective. I plan to build on my own
experience to create a technical framework, and a network of academic
and practitioner collaborators, to facilitate well-designed natural
experiments by the hosts of large online communities and a system for
data sharing that allows for academic evaluation of these experiments.

\emph{Understanding the Relationship Between Collective Action and
  Performance} -- My work has treated collective action and production
as ends in themselves and has largely avoided the consideration of
issues of performance, efficiency, and quality. Using my existing
datasets, I plan to compare the performance of collaborative
production to individually produced works to understand when
successful collection action leads to higher performance and
quality. In a manuscript currently under review using data from
Scratch, I show important limitations of collaboration in remixing
quality, particularly in regards to more artistic or media-intensive
works \cite{hill_cost_2012}. I will explore this direction in future
work.

\emph{Integrated Theory of Design for Collective Action} --
My studies of status provide a detailed understanding of the dynamics
of collection action in relation to one important independent
variable. In future work, I plan to evaluate the effect of governance
and different systems of authority, framing, modularity and project
complexity. In the long term, I hope to work toward a broad set of
principles of design for online collection action and community.

In graduate school, I have been fortunate to collaborate with many
co-authors in many academic departments and I intend to continue going
forward. In sum, my research uses design to contribute to social
scientific theories of collective action, and uses theories of
collective action to influence design. I believe my work offers
implications and opportunities for a broad range of disciplines and
fields.

% bibliography here
\renewcommand{\bibsection}{\section{\bibname}\prebibhook}
\baselineskip 14.2pt
\bibliography{refs-processed}
\bibliographystyle{unsrt}

\end{document}

